%%%%%%%%%%%%%%%%%%%%%%%%%%%%%%%%%%%%%%%%%%%%%%%%%%%%%%%%%%%%%%%%%%%%%%%%%%%%%%%%
%2345678901234567890123456789012345678901234567890123456789012345678901234567890
%        1         2         3         4         5         6         7         8

\documentclass[letterpaper, 10 pt, conference]{ieeeconf}  % Comment this line out
                                                          % if you need a4paper
%\documentclass[a4paper, 10pt, conference]{ieeeconf}      % Use this line for a4
                                                          % paper

\IEEEoverridecommandlockouts                              % This command is only
                                                          % needed if you want to
                                                          % use the \thanks command
\overrideIEEEmargins
% See the \addtolength command later in the file to balance the column lengths
% on the last page of the document

\usepackage[utf8]{inputenc}
\usepackage[T1]{fontenc}
\usepackage{import}
\usepackage{lettrine}
\usepackage{nameref}
\usepackage{amsmath,amssymb}
\usepackage{graphicx}
\usepackage{pdfpages}
\usepackage[margin=0.5in]{geometry} % for PAPER & MARGIN
\usepackage[many]{tcolorbox} % for COLORED BOXES (tikz and xcolor included)
\usepackage{amssymb}
\usepackage[utf8]{inputenc}
\newtheorem{definition}{Definition}[section]
\graphicspath{{./images/}}
\definecolor{main}{HTML}{DADADA}    % setting main color to be used
\definecolor{sub}{HTML}{F5F5F5}

\newtcolorbox{boxK}{
    sharpish corners, % better drop shadow
    boxrule = 0pt,
    toprule = 4.5pt, % top rule weight
    enhanced,
    fuzzy shadow = {0pt}{-2pt}{-0.5pt}{0.5pt}{black!35} % {xshift}{yshift}{offset}{step}{options} 
}

\newtcolorbox{boxF}{
    colback = sub,
    enhanced,
    boxrule = 1.5pt, 
    colframe = white, % making the base for dash line
    borderline = {0.8pt}{0pt}{main, dashed} % add "dashed" for dashed line
}


% The following packages can be found on http:\\www.ctan.org
%\usepackage{graphics} % for pdf, bitmapped graphics files
%\usepackage{epsfig} % for postscript graphics files
%\usepackage{mathptmx} % assumes new font selection scheme installed
%\usepackage{mathptmx} % assumes new font selection scheme installed
%\usepackage{amsmath} % assumes amsmath package installed
%\usepackage{amssymb}  % assumes amsmath package installed

\title{\LARGE \bf
A Multi-Account and Multi-Tenant Policy-Based Access Control (PBAC) Approach for Distributed Systems Augmented with Risk Scores Generation
}

%\author{ \parbox{3 in}{\centering Huibert Kwakernaak*
%         \thanks{*Use the $\backslash$thanks command to put information here}\\
%         Faculty of Electrical Engineering, Mathematics and Computer Science\\
%         University of Twente\\
%         7500 AE Enschede, The Netherlands\\
%         {\tt\small h.kwakernaak@autsubmit.com}}
%         \hspace*{ 0.5 in}
%         \parbox{3 in}{ \centering Pradeep Misra**
%         \thanks{**The footnote marks may be inserted manually}\\
%        Department of Electrical Engineering \\
%         Wright State University\\
%         Dayton, OH 45435, USA\\
%         {\tt\small pmisra@cs.wright.edu}}
%}

\author{Nicola Gallo$^{1}$ and Antonio Radesca$^{2}$% <-this % stops a space
\thanks{This research was conducted in 2023 by Nitro Agility S.r.l., Matera, Italy and it is licensed under CC BY-NC-ND 4.0. To view a copy of this license, visit https://creativecommons.org/licenses/by-nc-nd/4.0/deed.en.}% <-this % stops a space
\thanks{$^{1}$Nicola Gallo, Software Architect
        {\tt\small nicola.gallo at nitroagility.com}}%
\thanks{$^{2}$Antonio Radesca, Software Architect,
        {\tt\small antonio.radesca at nitroagility.com}}%
}

\begin{document}

\maketitle
\thispagestyle{empty}
\pagestyle{empty}

%%%%%%%%%%%%%%%%%%%%%%%%%%%%%%%%%%%%%%%%%%%%%%%%%%%%%%%%%%%%%%%%%%%%%%%%%%%%%%%%
\begin{abstract}
\import{./sections}{abstract.tex}
\end{abstract}

%%%%%%%%%%%%%%%%%%%%%%%%%%%%%%%%%%%%%%%%%%%%%%%%%%%%%%%%%%%%%%%%%%%%%%%%%%%%%%%%
\section{Introduction}
\label{sec:introduction}
\import{./sections}{introduction.tex}

\section{Problem Definition}
\label{sec:problemdefinition}
\import{./sections}{problem-definition.tex}

\newpage

\section{Policy-Based Approach}
\label{sec:permissioningapproach}
\import{./sections}{policy-based-approach.tex}

\section{RISK Scores Generation}
\label{sec:generatedriskscores}
\import{./sections}{generated-risk-scores.tex}

\section{conclusion}
\label{sec:conclusion}
\import{./sections}{conclusion.tex}

\addtolength{\textheight}{-12cm}   % This command serves to balance the column lengths
                                  % on the last page of the document manually. It shortens
                                  % the textheight of the last page by a suitable amount.
                                  % This command does not take effect until the next page
                                  % so it should come on the page before the last. Make
                                  % sure that you do not shorten the textheight too much.

%%%%%%%%%%%%%%%%%%%%%%%%%%%%%%%%%%%%%%%%%%%%%%%%%%%%%%%%%%%%%%%%%%%%%%%%%%%%%%%%

\begin{thebibliography}{99}
\bibitem{c1} Brewer, Eric. (2000). Towards robust distributed systems. PODC. 7. 10.1145/343477.343502. 
\bibitem{c2} V. Velepucha and P. Flores, "A Survey on Microservices Architecture: Principles, Patterns and Migration Challenges," in IEEE Access, vol. 11, pp. 88339-88358, 2023, doi: 10.1109/ACCESS.2023.3305687.
\bibitem{c3} V. Vernon, Domain-Driven Design Distilled. Reading, MA, USA: Addison-Wesley Professional, 2016.
\bibitem{c4} [OpenID.2.0] OpenID Foundation, “OpenID Authentication 2.0,” December 2007.
\bibitem{c5} [RFC6749]	M. Jones, D. Hardt, “The OAuth 2.0 Authorization Framework,” October 2012.
\bibitem{c6} [RFC6750]	M. Jones, D. Hardt, “The OAuth 2.0 Authorization Framework: Bearer Token Usage", October 2012
\bibitem{c7} L. Zhi, W. Jing, C. Xiao-su and J. Lian-xing, "Research on Policy-based Access Control Model," 2009 International Conference on Networks Security, Wireless Communications and Trusted Computing, Wuhan, China, 2009, pp. 164-167, doi: 10.1109/NSWCTC.2009.313.
\end{thebibliography}

\end{document}